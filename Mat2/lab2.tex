\input template.tex
\begin{document}
\selectlanguage{russian}
\maketitle 2 {АЛГЕБРАИЧЕСКОЕ ИНТЕРПОЛИРОВАНИЕ}
\setcounter{page}{2}
\normalfont
\ssec{Цель работы}
Цель работы заключается в том, чтобы изучить различные методы интерполяции и написать программу, реализующий один из таких методов.

\ 
\ssec{Задание на работу}
По значениям функции  f(x) построить  полином Ньютона с разделенными разностями.
 $$f(x)=x^4+3x-1$$

\ssec{Теоретическая справка}
При    проведении    эксперимента,  при    табулировании    сложных    функций    результат  
получают  в  виде  таблично-заданной  функции.  

Таблицы  делятся  на  два  вида:  регулярные (с равноотстоящими узлами)  и  нерегулярные:    
 $$x_i = x_0 + i_h .                      \eqno{1} $$

Величину  $h$  называют  шагом  таблицы.  У  нерегулярных  таблиц точки  по  оси  абсцисс  
размещаются  произвольно.  Точки  с  координатами $(x_i , y_i) , i = 0,1,…,n$  называют  узлами  
интерполяции.                           

Интерполяцию  функций  понимают  в  двух  значениях.   

В    узком    значении  под    интерполяцией    понимают    отыскание    величин    таблично  
заданной  функции,  соответствующих  промежуточным  (межузловым)  значениям  аргумента,  
отсутствующим  в  таблице .  

Под  интерполяцией  в  широком  смысле  понимают  отыскание  аналитического  вида  
функции  $y = F(x)$,  выбранной  из  определѐнного  класса  функций  и  точно  проходящую  
через  узлы  интерполяции .  При  этом  задача  формулируется  таким  образом :  на  отрезке  $[a 
,b]$   заданы $ (n+1)$  точки $ x_0 , x_1  , x_2 ,…, x_n $ и  значения некоторой  функции  $f(x)$  в  этих  
точках : 
               $$   f(x_0)= y_0 , f(x_1)= y_1 , … , f(x_n)= y_n . \eqno{(1)} $$ 

Вид    функции    либо    неизвестен    вовсе,  либо    он    неудобен    для    расчѐтов    (сложная  
функция,  которую  необходимо  при  расчѐтах  интегрировать,  дифференцировать  и  т. п.) .  

Требуется  построить  функцию  $F(x)$  (интерполирующую  функцию),  принимающую  в  узлах  
интерполяции  те  же  значения,  что  и  исходная  функция 
$f(x)$,  т. е. 
$$ F(x_0)= y_0 , F(x_1)= y_1 , … , F(x_n)= y_n .        \eqno{(2)}$$ 
 
В  такой  постановке  задача  имеет  бесконечное  множество  решений,  или  совсем  не  
имеет.  Однако  эта  задача  становится  однозначной,  если  вместо  произвольной  функции  
искать  полином  степени  не  выше  $n$, удовлетворяющий условиям (3): 
 
$$F(x) = a_0 + a_1 x + a_2 x^2 +…+ a_n x^n                        \eqno{(3)}$$ 
 
Такую    задачу    называют    алгебраической  или  полиномиальной  интерполяцией. 

Возможность  применения  полиномов  для  интерполяции    обосновывается  теоремами 
Вейерштрасса.  Применение полиномов для вычиcлений  представляет  определѐнные  удобства  
(при  интегрировании,  дифференцировании,  вычислении  значений    функции  в  силу  свойств  
аддитивности членов полинома и  простого вида каждого члена). 

Сущность  применения  интерполяционных  формул  состоит  в том, что  функция  $y=f(x)$,  
для    которой    известна    лишь    таблица    значений,  заменяется    интерполяционным  
многочленом,  который  рассматривается  как  приближѐнное  аналитическое  выражение  для  
функции  $f(x)$.  При  этом,  естественно,  возникает  вопрос о  точности такого приближения и 
оценки  погрешности, возникающей  при  замене  $f(x)$  на  $F(x)$ . 

Для  построения  интерполирующего  многочлена  применяются  многочисленные  способы интерполяции. Рассмотрим построение интерполяционных многочленов Ньютона с разделнными разностями.

Интерполяционным многочленом Ньютона называется многочлен 
\begin{align*}
P_n(x)&=f(x_0) + f(x_0,x_1)(x-x_0)+ f(x_0,x_1,x_2)(x-x_0)(x-x_1) + ...+ \cr
 			&+f(x_0,x_1,...,x_n)(x-x_0)(x-x_1)...(x-x_{n-1}), 
\end{align*}
где  $f (x_0,x _1),  f (x_0, x_1, x_2),…,  f (x_0,x_1,...,x_n)$   –  разделенные  разности,  которые  могут 
быть вычислены рекуррентно по формулам: 

Разделенная  разность  нулевого  порядка  функции  $f(x)$  –  сама  функция  $f(x)$. 

Разделенная  разность  n-го  порядка  определяется  через  разделенную  разность 
$(n-1)$-го порядка по формуле:  
  $$f (x_0,x_1,...,x_n)=  \frac{f (x_1,x_2,...,x_n)-f (x_0,x_1,...,x_{n-1})}{x_n-x_0}$$

\ssec{Схема алгоритма}
На рисунке \ref{newtonpoly} представлена схема алгоритма получения интерполяционного многочлена Ньютона с разделенными разностями.
\pic{LAB1.png}{Схема алгоритма  получения интерполяционного многочлена Ньютона}{newtonpoly}{H}
На рисунке \ref{getdiff} представлена схема алгоритма получения разделенных разностей для интерполяционного многочлена Ньютона.
\pic{LAB2.png}{Схема алгоритма  получения разделенных разностей}{getdiff}{H}
На рисунке \ref{multpoly} представлена схема алгоритма перемножения многочленов.
\pic{LAB3.png}{Схема алгоритма  перемножения многочленов}{multpoly}{H}
На рисунке \ref{cmultpoly} представлена схема алгоритма умножения многочлена на число.
\pic{LAB4.png}{Схема алгоритма умножения многочлена на число}{cmultpoly}{H}
На рисунке \ref{addpoly} представлена схема алгоритма сложения многочленов.
\pic{LAB5.png}{Схема алгоритма  сложения многочленов}{addpoly}{H}
%\vfill
\clearpage
\ssec{Инструкция пользователя}
Программа позволяет построить интерполяционный многочлен Ньютона с разделенными разностями.

Программе необходимо передать списки аргументов и значений интерполируемой функции. Каждый список можно создать 2-мя способами. Список аргументов можно полностью ввести с клавиатуры, указав перед этим количество элементов списка, а можно сгенерировать значениями из некоторого диапазона - в данном случае передайте программе границы интервала и шаг изменения аргумента. Список значений можно также ввести с клавиатуры для соответствующих аргументов и в том же порядке, в котором вводились аргументы, а можно рассчитать значения для введенных уже аргументов от тестовой функции $f(x)=x^4+3x-1$.

После завершения расчетов программа выведет на экран искомый интерполяционный многочлен.

\ 
\ssec{Инструкция программиста}
При разработке программы построение интерполяционного многочлена Ньютона с разделенными разностями были написаны следующие процедуры и функции:
%\suppressfloats[p]
\newcommand{\ftab}[1]{
функции представлены в таблице \ref{#1}
}
\elist{
\item make\_newton\_poly - функция расчета коэффициентов интерполяционного многочлена
      Ньютона с разделёнными разностями для таблично заданной функции.
      
Возвращает коэффициенты многочлена.
    
make\_newton\_poly(x,f)

Параметры \ftab{newtonpoly:1}:
\tabl{Параметры  функции расчета коэффициентов интерполяционного многочлена}{
    \tabln{x && список аргументов функции,}
    \tabln{f && список значений функции от аргументов х}
}{newtonpoly:1}{H}
    
Локальные переменные \ftab{newtonpoly:2}:
\tabl{Локальные переменные  функции расчета коэффициентов интерполяционного многочлена}{
    \tabln{d && список разделенных разностей}
    \tabln{res && коэффициенты многочлена Ньютона}
    \tabln{l && коэффициенты многочлена вида (x-x0)(x-x1)...}
}{newtonpoly:2}{H}
\item get\_diff - рекурсивная функция расчета разделенных разностей.
      
Возвращает список разделенных разностей.
  
get\_diff(x,i,k,f,l)
  
Параметры \ftab{getdiff:1}:
\tabl{Параметры функции расчета разделенных разностей}{
    \tabln{x && список аргументов функции,}
    \tabln{i && индекс разделенной разности}
    \tabln{k && степень разделенной разности}
    \tabln{f && список значений функции от аргументов х}
    \tabln{l && список разделенных разностей}
}{getdiff:1}{H}
\item mult\_poly - функция умножения многочленов.
      
Возвращает многочлен-произведение.
    
mult\_poly(l1,l2)

Параметры \ftab{multpoly:1}:
\tabl{Параметры  функции умножения многочленов}{
    \tabln{l1,l2 && списки коэффициентов полиномов-множителей}
}{multpoly:1}{H}
    
Локальные переменные \ftab{multpoly:2}:
\tabl{Локальные переменные   функции умножения многочленов}{
    \tabln{res && список коэффициентов полинома-произведения}
    \tabln{i,j && переменные счетчики для доступа к элементам списков}
}{multpoly:2}{H}
\item cmult\_poly - функция умножения многочлена на число.
      
Возвращает многочлен-произведение.
    
cmult\_poly(c,l)

Параметры \ftab{cmultpoly:1}:
\tabl{Параметры  функции умножения  многочлена на число}{
   \tabln{ c && константа-множитель,}
   \tabln{ l && список коэффициентов полинома-множителя}
}{cmultpoly:1}{H}
    
Локальные переменные \ftab{cmultpoly:2}:
\tabl{Локальные переменные   функции умножения  многочлена на число}{
   \tabln{ res && список коэффициентов полинома-произведения}
   \tabln{ i && переменная счетчик для доступа к элементам списка}
}{cmultpoly:2}{H}
\item add\_poly - функция суммирования многочленов.
      
Возвращает многочлен-сумму.
    
add\_poly(l1,l2)

Параметры \ftab{addpoly:1}:
\tabl{Параметры  функции суммирования многочленов}{
   \tabln{ l1,l2 && список коэффициентов полиномов-слагаемых}
}{addpoly:1}{H}
   
Локальные переменные  \ftab{addpoly:2}:
\tabl{Локальные переменные  функции суммирования многочленов}{
   \tabln{res && список коэффициентов полинома-суммы}
   \tabln{ i,j && переменные счетчики для доступа к элементам списков}
   \tabln{ m && максимальная из степеней многочленов-сумм}
}{addpoly:2}{H}
}
\clearpage
\ssec{Текст программы}
Реализиция задачи построения  интерполяционного многочлена Ньютона с разделенными разностями написана на языке Python 3.2 и состоит из двух частей.

Первая часть, файл mat.py, содержит вычислительное ядро и консольный интерфейс. Исходный текст этого модуля приводится ниже.

\prog{Python}{mat.py}

Вторая часть - файл inter.py - является графическим интерфейсом для вычислительного ядра первого модуля. Далее представлен текст этого второго модуля.

\prog{Python}{inter.py}

\clearpage

\ 
\ssec{Тестовый пример}
Ниже на рисунке \ref{scr:1} представлен пример работы программы при построении полинома Ньютона второй степени для функции $f(x)=x^4+3x-1$.
\pic{SCR1.png}{Пример работы программы}{scr:1}{H}
\clearpage
\ssec{Вывод}
В этой лабораторной работе я изучил различные методы интерполяции. Итерполяция необходима, когда из-за сложности исследуемой функции трудно провести её анализ, но допустимо взять другую, более простую функцию, которая проходит через некоторые точки исходной функции. Процесс получения такой более простой функции и называется интерполяцией. Интерполяция широко применяется в различных областях науки и техники.
\end{document}
