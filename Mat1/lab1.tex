\input template.tex
\begin{document}
\selectlanguage{russian}
\maketitle 1 {РЕШЕНИЕ СИСТЕМ ЛИНЕЙНЫХ АЛГЕБРАИЧЕСКИХ УРАВНЕНИЙ}
\setcounter{page}{2}
\normalfont
\ssec{Цель работы}
Цель работы заключается в том, чтобы изучить методы решения систем линейных алгебраических уравнений (СЛАУ) и написать программу, реализующий один из таких методов.

\ 
\ssec{Задание на работу}
Решить систему линейных алгебраических уравнений методом Гаусса-Зайделя.

Обеспечить ввод с клавиатуры и из файла (по выбору пользователя), а также
сохранение системы вместе с результатом в файл.

Обеспечить подсчет количества операций (сложения, вычитания, умножения
и деления по отдельности и в сумме), для итерационных методов – дополнительно
– количество потребовавшихся итераций и вывод на экран соответствующей
статистики.

Осуществлять контроль сходимости и выполнять проверку на существование
и единственность решения.
\vfill
\clearpage
\ssec{Теоретическая справка}
{\hbadness=10000
\loop{}\ifnum\thepage<5{\hrule\hfill\\}\repeat
\newcounter{lc}\setcounter{lc}{29}
\loop{}\ifnum\value{lc}>0{\addtocounter{lc}{-1}\hrule\hfill\\}\repeat
}
\ssec{Схема алгоритма}
На рисунке \ref{solvesys} представлена схема обобщенного алогритма решения системы линейных алгебраических уравнений.
\pic{LAB1.png}{Cхема обобщенного алогритма решения CЛАУ}{solvesys}{H}
На рисунке \ref{rank} представлена схема алогритма расчета рангов основной и расширенной матрицы СЛАУ.
\pic{LAB2.png}{Cхема алогритма расчета рангов}{rank}{H}
На рисунке \ref{gausstransform} представлена схема алогритма преобразования матрицы СЛАУ к верхнетрапециедальному виду с полным переупорядовачиванием.
\pic{LAB3.png}{Cхема алогритма преобразования матрицы СЛАУ}{gausstransform}{H}
На рисунке \ref{indexofmax} представлена схема алогритма нахождения индексов максимального элемента в подматрице (i,i,n,n) основной матрицы СЛАУ $A_{NxN}$.
\pic{LAB4.png}{Cхема алогритма нахождения индексов максимального элемента}{indexofmax}{H}
На рисунке \ref{ispositive} представлена схема алогритма проверки матрицы на диагональное преобладание.
\pic{LAB5.png}{Cхема алогритма проверки матрицы на диагональное преобладание}{ispositive}{H}
На рисунке \ref{atmatr} представлена схема алогритма домножения расширенной матрицы СЛАУ на транспонированную основную.
\pic{LAB6.png}{Cхема алогритма домножения расширенной матрицы СЛАУ}{atmatr}{H}
На рисунке \ref{zeidel} представлена схема алогритма решения СЛАУ по методу Гаусса-Зейделя.
\pic{LAB7.png}{Cхема алогритма решения СЛАУ по методу Гаусса-Зейделя}{zeidel}{H}
На рисунке \ref{zstep} представлена схема алогритма шага итерации по методу Гаусса-Зейделя.
\pic{LAB8.png}{Cхема алогритма шага итерации по методу Гаусса-Зейделя}{zstep}{H}
%\vfill
\clearpage
\ssec{Инструкция пользователя}
Программа позволяет решить систему линейных алгебраических уравнений с вещественными коэффициентами.

Приступая к работе, передайте программе расширенную матрицу системы, элементы уоторой разделяются пробелами. Ввод новой строки начинается с нажатия кнопки <Enter>. Матрица может быть не квадратной, но все строки должны быть одинаковой длины. Вместо отсутствующих членов уравнения пишется 0. После ввода матрицы передайте программе необходимую точность вычислений. Далее программа спросит о необходимости контроля сходимости метода на каждом шаге вычислений. Проверка сходимости позволяет предотвратить зависание программы, но в некоторых случаях является слишком строгим. Рекомендуется в первый раз запускать программу с включенным контролем сходимости, в случае отказа попробовать его отключить.

После ввода данных программа приступит к вычислениям. Программа определяет, если система не имеет решений или имеет их бесконечное множество, и выводит соответствующее сообщение на экран. Если система имеет единственное решение, программа выведет его в виде столбца $x_1,x_2...x_n$. Также программа выведет информацию о количестве итераций и числе использованных арифметических операций. По желанию, всю вышеперечисленную информацию программа может записать в файл.

\ 
\ssec{Инструкция программиста}
При разработке программы вычисления значения функции были написаны следующие процедуры и функции:
%\suppressfloats[p]
\newcommand{\ftab}[1]{
функции представлены в таблице \ref{#1}
}
\newcommand{\ftabI}[3]{
Параметры #1 представлены в таблице \ref{#2}, локальные переменные - в таблице \ref{#3}.
}
\newcommand{\ftabII}[3]{
Параметры-переменные #1 представлены в таблице \ref{#2}, локальные переменные - в таблице \ref{#3}.
}
\newcommand{\ftabIII}[4]{
Параметры-переменные #1 представлены в таблице \ref{#2}, параметры-значения - в таблице \ref{#3}, локальные переменные - в таблице \ref{#4}.
}
\elist{
\item{Функция solvesys - функция решения СЛАУ.
Возвращает - решение и статистику о решении.
Если решение не найдено, возвращает None и
поясняющее сообщение.
	
	solvesys(a,e,iter\_check=True)

Параметры \ftab{solvesys:1}:
	\tabl{Параметры  функции решения СЛАУ}{
	\tabln{a &список& расширенная матрица СЛАУ}
	\tabln{e &веществ.& точность вычислений}
	\tabln{iter\_check &булев.&\parbox[t]{10cm}{флаг необходимости проверки сходимости на каждом шаге итераций}}
	}{solvesys:1}{H}
Локальные переменные \ftab{solvesys:2}:
	\tabl{Локальные переменные функции решения СЛАУ}{
	\tabln{at &список& \parbox[t]{10cm}{матрица системы, приведенная к трапец. виду
		(с полным переупорядовачением)}}
	\tabln{r,rext &целое& ранги основной и расширенной матриц СЛАУ}
	}{solvesys:2}{H}
}
\item Функция gauss\_transform - функция
    преобразования расширенной матрицы СЛАУ
    с полным переупорядовачиванием.
    Возвращает преобразованную матрицу.

        gauss\_transform(a)
        
Параметры \ftab{gausstransform:1}:
	\tabl{Параметры функции
    преобразования расширенной матрицы СЛАУ}{
    \tabln{a &список& расширенная матрица СЛАУ}
	}{gausstransform:1}{H}

Локальные переменные \ftab{gausstransform:2}:
	\tabl{Локальные переменные  функции
    преобразования расширенной матрицы СЛАУ}{
    \tabln{at &список& копия матрицы а для преобразования.}
    \tabln{i,j,k &целое& переменные-счетчики для перебора элементов матрицы a.}
    \tabln{i2,j2 &целое& \parbox[t]{12cm}{координаты элемента для перестановки на диагональ
        матрицы}}
    \tabln{factor &веществ.& коэффициент домножения на слагаемую строку матрицы}
	}{gausstransform:2}{H}
\item index\_of\_max - функция  нахождения индексов максимального по модулю элемента 
в подматрице (i,i,n,n) в основной матрице СЛАУ A(n x n).
		
Внутренняя функция функции gauss\_transform.
			
			index\_of\_max()
		
Переменные объемлющей функции \ftab{indexofmax:1}:
\tabl{Переменные объемлющей функции функции  нахождения индексов максимального по модулю элемента}{
	\tabln{at &список& копия матрицы а для преобразования.}
	\tabln{i &целое& переменная-счетчик; кордината левого верхнего угла подматрицы}
	}{indexofmax:1}{H}
	
Локальные переменные \ftab{indexofmax:2}:
\tabl{Локальные переменные функции нахождения индексов максимального по модулю элемента}{
	\tabln{m &веществ.& модуль максимального элемента}
	\tabln{j,k &целое& переменные-счетчики для перебора элементов матрицы at.}
	\tabln{i2,j2 &целое& индексы максимального элемента}
	}{indexofmax:2}{H}
	
\item Функция is\_positive - функции проверки матрицы на диагональное преобладание.
    Если матрица имееет диагональное преобладание, возвращает True,
    иначе возвращает False.
	
		is\_positive(a)
		
Параметры \ftab{ispositive:1}:
	\tabl{Параметры функции проверки матрицы на диагональное преобладание}{
    \tabln{a &список& расширенная матрица СЛАУ}
	}{ispositive:1}{H}

Локальные переменные \ftab{ispositive:2}:
	\tabl{Локальные переменные функции проверки матрицы на диагональное преобладание}{
    \tabln{k &целое& количество преобладающих диагональных элементов}
    \tabln{s &веществ.& сумма модулей элементов строки без диагонального.}
    \tabln{i,j &целое& переменные-счетчики для перебора элементов матрицы a.}
	}{ispositive:2}{H}

\item rank - функция расчета рангов основной и расширенной матриц СЛАУ.
		
		rank(a)

Параметры \ftab{rank:1}:
	\tabl{Параметры функции расчета рангов основной и расширенной матриц}{
    \tabln{a &список& расширенная матрица СЛАУ}
	}{rank:1}{H}

Локальные переменные \ftab{rank:2}:
	\tabl{Локальные переменные функции расчета рангов основной и расширенной матриц}{
    \tabln{at &список& преобразованная в трапециедальную матрица а}
    \tabln{r,rext &целое& ранги основной и расширенной матриц СЛАУ}
    \tabln{i &целое& переменная-счетчик для перебора элементов матрицы a.}
	}{rank:2}{H}

\item at\_matr - функция домножения расширенной матрицы СЛАУ на транспонированную основную.
    
		at\_matr(a)
	
Параметры \ftab{atmatr:1}:
	\tabl{Параметры функции домножения расширенной матрицы на транспонированную}{
    \tabln{a &список& расширенная матрица СЛАУ}
	}{atmatr:1}{H}

Локальные переменные \ftab{atmatr:2}:
	\tabl{Локальные переменные функции домножения расширенной матрицы на транспонированную}{
    \tabln{an &список& преобразованная матрица СЛАУ}
    \tabln{i,j,k &целое& переменные-счетчики.}
	}{atmatr:2}{H}

\item norm - функция расчета нормы матрицы $\alpha$ для методов
    простых итераций и Гаусса-Зейделя.
    	
		norm(a)
	
Параметры \ftab{norm:1}:
	\tabl{Параметры функции расчета нормы матрицы}{
    \tabln{a &список& расширенная матрица СЛАУ}
	}{norm:1}{H}

Локальные переменные \ftab{norm:2}:
	\tabl{Локальные переменные функции расчета нормы матрицы}{
    \tabln{s &веществ.& сумма квадратов строки матрицы $\alpha$}
    \tabln{line &список& строка матрицы alpha}
    \tabln{a\_ij &веществ.& текущий элемент alpha} 
    \tabln{i,j &целое& переменные-счетчики.}
	}{norm:2}{H}

\item zeidel - функция решения СЛАУ по методу Гаусса-Зейделя
    Возвращает - решение и статистику о решении.
    Если решение не найдено, возвращает None и
    поясняющее сообщение.
	
		zeidel(a,e,iter\_check)
		
Параметры \ftab{zeidel:1}:
	\tabl{Параметры функции решения СЛАУ по методу Гаусса-Зейделя}{
    \tabln{a &список& расширенная матрица СЛАУ}
    \tabln{e &веществ.& точность вычислений}
	\tabln{iter\_check &булев.&\parbox[t]{10cm}{ флаг необходимости проверки сходимости
		на каждом шаге итераций}}
	}{zeidel:1}{H}

Локальные переменные \ftab{zeidel:2}:
	\tabl{Локальные переменные функции решения СЛАУ по методу Гаусса-Зейделя}{
	\tabln{x &список& решение СЛАУ}
	\tabln{stats &словарь& статистика решения}
	\tabln{n &веществ.& коэффициент оценки погрешности}
	\tabln{ek0,ek &веществ.& точность на предыдущем и текущем шаге}
	}{zeidel:2}{H}

\item z\_step - функция итерации метода Гаусса-Зейделя. Внутрення функция функции zeidel.
	Возвращает рассчитанное решение и точность этого решения.

		z\_step()
			
Переменные объемлющей функции(zeidel) \ftab{zstep:1}:
	\tabl{Переменные объемлющей функции функции итерации метода Гаусса-Зейделя}{
	\tabln{a &список& расширенная матрица СЛАУ}
	\tabln{x &список& решение СЛАУ}
	\tabln{stats &словарь& статистика решения}
	}{zstep:1}{H}
Локальные переменные \ftab{zstep:2}:
	\tabl{Локальные переменные функции итерации метода Гаусса-Зейделя}{
	\tabln{ek &веществ.& точность на текущем шаге}
	\tabln{i,j &целое& переменные-счетчики для перебора элементов матрицы a.}
	}{zstep:2}{H}
}
\clearpage
\ssec{Текст программы}
Ниже представлен текст программы на языке Python 3.2, реализующей метод решения СЛАУ Гаусс-Зейделя.
%\prog{Python}{mat.py}
\clearpage

\ 
\ssec{Тестовый пример}
Ниже на рисунке представлен пример работы программы для системы, имеющей бесконечно много решений.
\pic{SCR1.png}{Пример работы программы для несовместной системы}{scra}{H}

На рисунке представлен пример работы программы для системы, имеющей единственное решение.
\pic{SCR2.png}{Пример работы программы для совместной системы}{scrb}{H}
\ 
\ssec{Вывод}
В этой лабораторной работе я изучил различные методы решения СЛАУ. Такие системы часто встречаются в различных областях науки и хозяйства - математике, физике, химии, экономике, однако решение систем вручную достаточно сложная и утомительная операция. Поэтому необходимо знать численные методы решения СЛАУ, и уметь реадизовать их на практике.
\end{document}
