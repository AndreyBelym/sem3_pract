\input template.tex
\begin{document}
\selectlanguage{russian}
\maketitle 4 {ЧИСЛЕННОЕ ИНТЕГРИРОВАНИЕ}
\setcounter{page}{2}
\normalfont
\ssec{Цель работы}
Цель работы заключается в том, чтобы изучить различные методы численного интегрирования и написать программу, реализующий один из таких методов.

\ 
\ssec{Задание на работу}
Вычислить значение интеграла методом средних прямоугольников:
 $$\int_a^b\frac{x}{1+x^2}dx$$

\ssec{Теоретическая справка}
Пусть   $f(x)$ - функция, непрерывная и достаточно гладкая на отрезке $[a,b]$ . 

Требуется найти интеграл $I=\int_a^b f(x) dx$. 

Численные методы обычно применяются при вычислении интегралов, которые не берутся, или имеют сложные подынтегральные функции.

Все численные методы строятся на том, что подынтегральная функция приближенно заменяется более простой (горизонтальной или наклонной прямой, параболой 2-го, 3-го или более высокого порядка), от которой интеграл легко вычислить. В результате получаются формулы интегрирования, называемые квадратурными, в виде взвешенной суммы ординат подынтегральной функции в отдельных точках:
 
$\int_a^b f(x) dx\approx \sum_i \omega_i f(x_i)$

Чем меньше интервалы, на которых производят замену, тем точнее вычисляется интеграл. Поэтому исходный отрезок $[а,b]$ для повышения точности делят на несколько равных или неравных интервалов, на каждом из которых применяют формулу интегрирования, а затем складывают результаты.

Все методы различаются значениями ординат $x_i$  и весов $\omega_i$ . 
В большинстве случаев погрешность численного интегрирования определяется путем двойного интегрирования: с исходным шагом (шаг определяется путем равномерного деления отрезка $b - a$ на число отрезков $n$:$ h = (b - a) / n$ и с шагом, увеличенным в 2 раза. Разница вычисленных значений интегралов определяет погрешность.

Сравнение эффективности различных методов проводится по степени полинома, который данным методом интегрируется точно, без ошибки. Чем выше степень такого полинома, тем выше точность метода, тем он эффективнее. 

К простейшим методам можно отнести методы прямоугольников (левых, правых и средних) и трапеций. В первом случае подынтегральная функция заменяется горизонтальной прямой ($у =с_0$) со значением ординаты (т.е. значением функции) соответственно слева, справа или посередине участка, во втором случае — наклонной прямой ($у=с_1х+с_0$).
Формулы интегрирования при разбиении отрезка $[a,b]$ на $n$ частей с равномерным шагом $h$ соответственно приобретают вид:

• для одного участка интегрирования (простые формулы):

- метод прямоугольников
  
$I=\int_a^b f(x) dx\approx f(a) h$,

$I=\int_a^b f(x) dx\approx f(a+h/2) h$,
 
$I=\int_a^b f(x) dx\approx f(b) h$ ,

$\omega_0=h=b-a$ ;

- метод трапеций
 $I=\int_a^b f(x) dx\approx \frac{f(a)+f(b)}h$ ,

$\omega_0=\frac{b-a}2 $,

• для n участков интегрирования (составные формулы):

- метод прямоугольников
 $I=\int_a^b f(x) dx\approx \sum_{i=0}{n-1}\omega_i f(x_i)$,
 
 $I=\int_a^b f(x) dx\approx \sum_{i=0}{n-1}\omega_i f(x_i+h/2)$,

 $I=\int_a^b f(x) dx\approx \sum_{i=1}{n}\omega_i f(x_i)$,

$\omega_i=h=const$.

- метод трапеций
 
$I=\int_a^b f(x) dx\approx \sum_{i=1}{n}\omega_i f(x_i)$,

$\omega_i=h/2=const$.

Нетрудно заметить, что в методе прямоугольников интеграл вычислится абсолютно точно только при $f(x) = c $, а в методе трапеций — при $f(x)$ линейной или кусочно-линейной.

Метод прямоугольников не находит практического применения в силу значительных погрешностей

\ssec{Схема алгоритма}
На рисунке \ref{LAB1} представлена схема алгоритма расчета определенного интеграла методом средних прямоугольников
до достижения указанной точности.
\pic{LAB1.png}{Схема алгоритма расчета интеграла до достижения точности}{LAB1}{H}
На рисунке \ref{LAB2} представлена схема алгоритма  расчета определенного интеграла методом средних прямоугольников
при указанном количестве шагов.
\pic{LAB2.png}{Схема алгоритма  расчета  интеграла при указанном количестве шагов}{LAB2}{H}

%\vfill
\clearpage
\ssec{Инструкция пользователя}
Программа позволяет провести расчет определенного интеграла методом средних прямоугольников
до достижения указанной точности.

Для работы введите параметры расчета (границы интервала интегрирования и требуюмую точность) в соотвествующие поля ввода на форме и нажмите кнопку "Посчитать". Программа выведет значение определенного интеграла на форму.

\ 
\ssec{Инструкция программиста}
При разработке программы численного интегрирования методом средних прямоугольников  были написаны следующие функции:
%\suppressfloats[p]

\elist{
\item rectIntegral - расчет определенного интеграла методом средних прямоугольников
при указанном количестве шагов.

Возвращает рассчитанное значение определенного интеграла.

rectIntegral::(Double->Double)->(Double,Double)->Double->Double

Параметры  функции \ftab{rectIntegral:1}:
\tabl{Параметры  функции  расчета определенного интеграла при указанном количестве шагов}{
\tabln{f & (Double->Double) & интегрируемая функция}
\tabln{ (a,b) & (Double,Double) & границы интегрирования}
\tabln{n & Double & число шагов изменения аргумента функции}
}{rectIntegral:1}{H}

Локальные определения  функции \ftab{rectIntegral:2}:
\tabl{Локальные определения  функции  расчета определенного интеграла при указанном количестве шагов}{
\tabln{h & Double & шаг изменения аргумента функции}
}{rectIntegral:2}{H}

\item firstWhenEps - возвращает первый элемент списка, для которого 
выполнено условие точности.

firstWhenEps::[Double]->Double->Double

Параметры  функции \ftab{firstWhenEps:1}:
\tabl{Параметры  функции выбора элемента списка}{
\tabln{xs & [Double] & список приближений}
\tabln{ eps & Double & значение точности}
}{firstWhenEps:1}{H}

\item rectIntegrate - расчет определенного интеграла методом средних прямоугольников
до достижения указанной точности.

rectIntegrate::(Double->Double)->(Double,Double)->Double->Double 

Параметры  функции \ftab{rectIntegrate:1}:
\tabl{Параметры  функции  расчета определенного интеграла до достижения точности}{
\tabln{ f & (Double->Double) & интегрируемая функция}
\tabln{(a,b) & (Double,Double) & границы интегрирования}
\tabln{eps & Double & значение точности}
}{rectIntegrate:1}{H}

Локальные определения  функции \ftab{rectIntegrate:2}:
\tabl{Локальные определения  функции  расчета определенного интеграла до достижения точности}{
\tabln{n0 & Double & первоначальное количество шагов}
}{rectIntegrate:2}{H}
}
\clearpage
\ssec{Текст программы}
Реализиция задачи численного интегрирования методом средних прямоугольников написана на языке Haskell 98 и состоит из двух частей.

Первая часть является вычислительным ядром программы, текст этой части приводится ниже:

\prog{Haskell}{MyIntegral.hs}

Во втрой части содержится реализация графического интерфейса:

\prog{Haskell}{Inter.hs}

\clearpage

\ 
\ssec{Тестовый пример}
Ниже на рисунке \ref{SCR1} представлен пример работы  численного интегрирования методом средних прямоугольников $\int_0^1\frac{x}{1+x^2}dx$.
\pic{SCR1.png}{Пример работы программы численного интегрирования}{SCR1}{H}
\clearpage
\ssec{Вывод}
В этой лабораторной работе я изучил различные методы численного интегрирования. Эти методы основаны на геометрическом смысле определенного интеграла. Самым простым методом интегрирования является метод прямоугольников. Численное интегрирование требуется при решении различных прикладных задач.
\end{document}
