\input template.tex
\begin{document}
\selectlanguage{russian}
\maketitle {МЕТОД ЭЙЛЕРА ДЛЯ РЕШЕНИЯ СИСТЕМ ДИФФЕРЕНЦИАЛЬНЫХ УРАВНЕНИЙ}
\setcounter{page}{2}
\normalfont
\addcontentsline{toc}{section}{Содержание}
\tableofcontents
\clearpage
\ssec{Общие сведения о дифференциальных уравнениях}
Дифференциальные  уравнения  связаны  с  построением  моделей  динамики  (движения) 
объектов  исследования.  Они  описывают,  как  правило,  изменение  параметров  объектов  во 
времени  (хотя  могут  быть  и  другие  случаи).  Результатом  решения  дифференциальных 
уравнений являются функции, а не числа, как при решении алгебраических уравнений, поэтому 
они и более трудоемки. 

При  использовании  численных  методов  решение  дифференциальных  уравнений 
представляется в табличном виде, т.е. получается совокупность значений  $(X_n,Y_n)$. Решение 
носит шаговый характер, т.е. по одной или нескольким начальным точкам  $(X,Y)$  за один шаг 
находят  следующую  точку,  затем  следующую  и  т.д.  Решение  между  двумя  соседними 
значениями аргумента $h= X_{n+1}-X_n$ называется шагом.  

Однако  прежде  чем  обсуждать  методы  решения,  приведем  некоторые  сведения  из  курса 
дифференциальных уравнений. 

В зависимости от числа независимых переменных, дифференциальные уравнения делятся 
на  две  категории:  обыкновенные  дифференциальные  уравнения  \\
(ОДУ),  содержащие  одну 
независимую  переменную,  и  уравнения  с  частными  производными, содержащими  нес\-колько 
независимых переменных (например, в механике сплошных сред искомой функцией является 
плотность,  $t^0$ ,  напряжение  и  др.,  а  аргументами  -  координаты  рассматриваемой  точки  в 
пространстве и время). 
 
     Обыкновенные  дифференциальные  уравнения  могут  содержать  одну  или  нес\-колько 
производных от искомой функции  $y = f (x)$ и могут быть записаны в виде:   
                                           $$ F (x, y, y ',..., y^{(n)}) = 0,      \eqno{(1)}$$ 
где   x  – независимая переменная. 
     
 Наивысший  порядок  $(n)$  производной,  входящей  в  уравнение  (1)  называется 
порядком дифференциального уравнения. В частности\\ 
  $F(x, y, y') = 0$ - дифференциальное уравнение I порядка.\\ 
  $F(x, y, y', y'') = 0$ - дифференциальное уравнение II порядка.\\ 
В ряде случаев удается выразить старшую производную в явном виде  
$$ y'=f(x,y );   y''=f(x , y ,y ').$$ 

Такие  уравнения  называют  уравнениями,  разрешенными  относительно  старшей 
производной. 

Линейным  дифференциальным уравнением  называется  уравнение,  линейное  относительно 
искомой функции и её производных.  

Системой дифференциальных уравнений первого порядка называют систему вида 
$$\left\{\begin{array}{cc}
\frac{dy_1}{dx} & =f_1(x,y_1..y_n)\\
\frac{dy_2}{dx} & =f_2(x,y1_1..y_n)\\
...&...\\
\frac{dy_n}{dx} & =f_n(x,y_1..y_n)
\end{array}\right.
$$
  Решением дифференциального уравнения (1) $ n$ -го порядка называется всякая функция 
$y=\phi(x)$, которая после ее подстановки в (1) превращает его в тождество. Решение ОДУ 
может быть общим и частным. 

Общее решение ОДУ n -го порядка содержит  n  произвольных постоянных \\
$C_1 , C_2 ,C_3,..., C_n $ , т.е. решение ОДУ имеет вид:  $y =\phi(x,C_1,C_2,...,C_n)$. 
         
Частное решение ОДУ получается из общего, если  произвольным постоянным задать 
определенные значения. 
 
\ssec{Метод Эйлера для решения дифференциальных уравнений} 
Будем искать решение на ряде дискретных точек  $t_0,t _1,...,t_n$, удаленных друг от друга на 
расстоянии $h = t_{n+1}- t_n=const$ , в виде 
                                     
$$x(t_1)=x(t_0)+\int_{t_0}^{t_1}f(x,t)dt,$$
полученном путем интегрирования уравнения  $dx=f(x,t)dt$ . 

Если принять, что на отрезке $[t0,t1]\ x' = x'(t_0) = f (x_0,t_0) = const$ , то 
$$ x(t_1)=x(t_0)+f(x,t)(t_1-t_0)$$
или, обозначив $t_1-t_0=h$, в дискретном виде  
$$x_1=x_0+x'_0h.$$
Для точки  $x_{n+1}$ можно записать  
                                             $$  x_{n+1}= x_n +x'_nh.  $$ 
Полученное выражение известно как явный (прямой) метод Эйлера. 

Искомая  функция  $x(t)$  на  шаге  интегрирования  была  аппроксимирована  прямой, 
совпадающей с касательной в точке  $x_n=x (t_n)$.  
 
В указанном выражении производная вычислялась в точке  $(x_0,t_0)$. Можно также выразить  $x_1$ 
через  $x_0$  и производную в точке $(x_1,t_1)$, т.е.  $x'_1 = f (x_1,t_1)$. Тогда  получим 
$$x_1=x_0+x'_1h$$
Или в общем виде 
                                          $$x_{n+1} =x_n+x'_{n+1} h $$ 
Эта формула называется неявным (обратным) методом Эйлера .
 
Последнюю формулу можно представить  в виде $x_{n+1} =x_n +f(x_{n+1},t_{n+1} )h$, где  $x_{n+1}$ входит и в правую 
часть .  Поэтому  эта  формула  пригодна,  когда  будет  предсказано  значение  $x_{n+1}$, 
например,  с  помощью  явного  метода  Эйлера.  Таким  образом,  мы  пришли  к  понятию 
«предсказание»,  когда  определяется  значение  искомой  функции  в  последующей  точке.  На 
основе  найденного  «предсказания»  можно  рассчитать  значение  $x'_{n+1}=  f ( x_{n+1},t_{n+1} )$ и 
использовать  его  при  коррекции,  которую  выполним  по  неявной  формуле  Эйлера.  Из-за 
ошибки «предсказания» может быть получена неточная коррекция. Чаще всего «предсказание» 
используется  в  качестве  начального  приближения  для  решения  уравнения  методом 
Ньютона. 
         
Еще  одну  формулу  численного  интегрирования  можно  получить,  приняв \\
$f(x,t)=\frac{1}{2}(x'_n+x'_{n+1})$, тогда:                                                                                        
$$x_{n+1}=x_n+\frac{1}{2}(x'_n+x'_{n+1}).$$

Это формула трапеции, которую иногда называют модифицированным методом Эйлера. 

Это  также  неявная  формула  интегрирования,  т.  к.  неизвестная  величина  $x'_{n+1}$  входит  в 
правую часть. Значение переменной  $x_{n+1}$ получают из решения нелинейного алгебраического 
уравнения  
$$x_{n+1}=x_n+\frac{1}{2}(f(x_n,t_n)+f(x_{n+1},t_{n+1})$$
методом Ньютона. 

Алгоритм неявного метода Эйлера отличается от алгоритма метода трапеции отсутствием в 
формуле определения  $x$  составляющей  $f (x_0,t_0)$  и вместо 
$\frac{1}{2}h$  используется $h$ . 

\ssec{Метод Эйлера для решения систем дифференциальных уравнений} 
Метод Эйлера легко обобщается для случая системы обыкновенных дифференциальных уравнений (первой степени).

Пусть дана система 
$$\left\{\begin{array}{cc}
\frac{dy_1}{dx} & =f_1(x,y_1..y_n)\\
\frac{dy_2}{dx} & =f_2(x,y_1..y_n)\\
...&...\\
\frac{dy_n}{dx} & =f_n(x,y_1..y_n)
\end{array}\right.
.$$

Интегрируя обе части уравнений системы и перенося $y_{01},y_{02},..,y_{0n}$ в правые части, получаем:
$$\left\{\begin{array}{cc}
y_{11} & = y_{01}+\int_{x0}^{x1} f_1(x,y_1..y_n)\\
y_{12} & = y_{02}+\int_{x0}^{x1} f_2(x,y_1..y_n)\\
...&...\\
y_{1n} & = y_{0n}+\int_{x0}^{x1} f_n(x,y_1..y_n)
\end{array}\right.
$$

Считая, что функции $ f_1(x,y_1..y_n)=f_1(x_0,y_{01}..y_{0n})=const$,\\ $ f_2(x,y_{01}..y_{0n})=f_2(x_0,y_{01}..y_{0n})=const$, .. ,$f_n(x,y_{01}..y_{0n})=\\=f_n(x_0,y_{01}..y_{0n})=const$ на отрезке $[x_0,x_1]$, получаем:

$$\left\{\begin{array}{cc}
y_{11} & = y_{01}+f_1(x_0,y_{01}..y_{0n})(x_1-x_0)\\
y_{12} & = y_{02}+f_2(x_0,y_{01}..y_{0n})(x_1-x_0)\\
...&...\\
y_{1n} & = y_{0n}+f_n(x_0,y_{01}..y_{0n})(x_1-x_0)
\end{array}\right.,
$$ или в матричной форме, принимая $h=x_1-x_0$:
$$Y_1=Y_0+F_0\cdot h.$$

Cоответственно, для узла $x_{n+1}$ формула имеет вид:
$$\left\{\begin{array}{cc}
y_{(i+1)1} & = y_{i1}+f_1(x_i,y_{i1}..y_{in})(x_{i+1}-x_i)\\
y_{(i+1)2} & = y_{i2}+f_2(x_i,y_{i1}..y_{in})(x_{i+1}-x_i)\\
...&...\\
y_{(i+1)n} & = y_{in}+f_n(x_i,y_{i1}..y_{in})(x_{i+1}-x_i)
\end{array}\right.,
$$ или в матричной форме:
$$Y_{i+1}=Y_i+F_i \cdot h.$$

Однако следует учитывать, что данный метод даёт лишь очень приближенные результаты - его погрешность составляет величину порядка $(x_1-x_0)^n$.
\clearpage
\ssec{Вывод}
Метод Эйлера — наиболее простой численный метод решения обыкновенных дифференциальных уравнений. Метод Эйлера является  одношаговым методом первого порядка точности, основанном на аппроксимации интегральной кривой кусочно- линейной функцией.

Метод имеет невысокую точность (порядка расстояния между узлами $h$ в случае единственного уравнения и $h^n$ в случае системы уравнений, где $n$ - число уравнений системы) и характеризуется вычислительной неустойчивостью, поэтому для практического нахождения решений задачи Коши метод Эйлера применяется редко. Однако в виду своей простоты метод Эйлера находит свое применение в теоретических исследованиях дифференциальных уравнений, задач вариационного исчисления и ряда других математических проблем.
\clearpage
\ssecO{Список литературы}
\vspace{-1.2cm}
\elist{\item Киреев  В.И.  Пантелеев  А.В.  Численные  методы  в  примерах  и  задачах:  Учеб.  пособие. --- 3-е изд. стер. ---   М. 
Высш. шк., 2008.
\item Н.С.Бахвалов, Н.П.Жидков, Г.М.Кобельков. Численные методы.\ ---\\--- М.: Наука, 2001.
\item Интернет-ресурс "Википедия - свободная энциклопедия"\ ---\\--- http://ru.wikipedia.org/wiki/Метод\_Эйлера
}
\end{document}
