\input template.tex
\begin{document}
\selectlanguage{russian}
\maketitle 3 {СРЕДНЕКВАДРАТИЧЕСКОЕ ПРИБЛИЖЕНИЕ ФУНКЦИЙ}
\setcounter{page}{2}
\normalfont
\ssec{Цель работы}
Цель работы заключается в том, чтобы изучить  методы среднеквадратического приближения и написать программу, реализующий один из таких методов.

\ 
\ssec{Задание на работу}
Выполнить среднеквадратическое приближение  по тригонометрическому базису для функции:
 $$f(x)=x^4+3x-1$$

\ssec{Теоретическая справка}

Пусть значения приближаемой функции  $f(x)$ заданы в $N+1$  узлах: $f(x_0),...,f(x_N)$  . 

Аппроксимирующую функцию будем выбирать из некоторого параметрического семейства  $F(x,\vec{c})$, где 

$\vec{c}=(c_0,c_1,...,c_n)^T$   - вектор параметров, $N>n$ .

Принципиальным отличием задачи среднеквадратического приближения от задачи интерполяции является то, что число узлов превышает число параметров. В данном случае практически всегда не найдется такого вектора параметров, для которого значения аппроксимирующей функции совпадали бы со значениями аппроксимируемой функции во всех узлах.

В этом случае задача аппроксимации ставится как задача поиска такого вектора параметров  $\vec{c}=(c_0,c_1,...,c_n)^T$, при котором значения аппроксимирующей функции $F(x,\vec{c})$  как можно меньше отклонялись бы от значений аппроксимируемой функции в совокупности всех узлов.

Можно записать различные критерии близости аппроксимируемой и аппроксимирующей функции. 

Наиболее известными и часто используемыми являются 

критерий равномерного приближения:

 $$J(\vec{c})=\sum_{i=0}^N (f(x_i)-F(x,\vec{c})\to \min						\eqno{(1)}$$

критерий среднеквадратического приближения (метод наименьших квадратов):

 $$J(\vec{c})=\sqrt{\sum_{i=0}^N (f(x_i)-F(x,\vec{c})^2}\to \min						\eqno{(2)}$$

В обоих случаях коэффициенты многочлена выбирают исходя из условия  

$$\vec{c}=arg_{\vec{c}} min J(\vec{c})$$.

Подкоренное выражение в функции (2) представляет собой квадратичную функцию относительно коэффициентов аппроксимирующего многочлена. Она непрерывна и дифференцируема по  $c_0,c_1,...,c_n$. Очевидно, что ее минимум находится в точке, где все частные производные равны нулю

 $\frac{\partial{J(\vec{c})}}{\partial{c_m}}, m=0,1...,n$ . 

Приравнивая к нулю частные производные, получим систему линейных алгебраических уравнений относительно неизвестных (искомых) коэффициентов многочлена наилучшего приближения.

Метод наименьших квадратов может быть применен для различных параметрических функций, но часто в инженерной практике в качестве аппроксимирующей функции используются многочлены по какому-либо линейно независимому базису $\{\phi_k(x) , k=0,1,...,n\}$: 

$F(x,\vec{c})=\sum_{k=0}n c_k \phi_k(x)$ . 

В этом случае СЛАУ для определения коэффициентов будет иметь вполне определенный вид:
$$\left\{{\begin{array}{cccccccc}
c_1 a_00&+&c_2 a_01&+&...&+&c_n a0_n&=b_1\\
c_1 a_10&+&c_2 a_11&+&...&+&c_n a1_n&=b_2\\
&...&&&...&&&...\\
c_1 a_n0&+&c_2 a_n1&+&...&+&c_n an_n&=b_n\\
\end{array}}\right.$$,\\
$a_kj=\sum_{i=0}N \phi_k(x_i) \phi_j(x_i), b_j=\sum_{i=0}{N} \phi_j(x_i) f(x_i)$ .   

Чтобы эта система имела единственное решение необходимо и достаточно, чтобы определитель матрицы А (определитель Грама) был отличен от нуля. Для того, чтобы система имела единственное решение необходимо и достаточно чтобы система базисных функций $\phi_k(x) , k=0,1,...,n$  была линейно независимой на множестве узлов аппроксимации.
Очень распространено среднеквадратическое приближение многочленами по степенному базису $\{ x_k, k=0,1,...,n\}$ и по тригонометрическому 

$\{\phi_0(x)=1, \phi_1(x)=\sin(x), \phi_2(x)=\cos(x), \phi_3(x)=\sin(2x), \phi_4(x)=\cos(2x), ...        \}$ или 

$\phi_k=\left\{
{\begin{array}{ll}
\cos{(nx)},& k=2n,\\
\sin{((n+1)x)},&k=2n+1\\
\end{array}
}\right. k=0,1,...,N.$
\ssec{Схема алгоритма}
На рисунке \ref{LAB01} представлена схема алгоритма расчета коэффициентов аппроксимирующей функции на основе тригонометрического базиса.
\pic{LAB01.png}{Схема алгоритма  расчета коэффициентов аппроксимирующей функции}{LAB01}{H}
На рисунке \ref{LAB02} представлена схема алгоритма расчета значения члена тригонометрического базиса.
\pic{LAB02.png}{Схема алгоритма  расчета члена тригонометрического базиса}{LAB02}{H}
На рисунке \ref{LAB03} представлена схема алгоритма получения матрицы СЛАУ  коэффициентов аппроксимирующей функции на основе тригонометрического базиса.
\pic{LAB03.png}{Схема алгоритма получения матрицы СЛАУ}{LAB03}{H}
На рисунке \ref{solvesys} представлена схема обобщенного алгоритма решения системы линейных алгебраических уравнений.
\pic{LAB1.png}{Cхема обобщенного алгоритма решения CЛАУ}{solvesys}{H}
На рисунке \ref{rank} представлена схема алгоритма расчета рангов основной и расширенной матрицы СЛАУ.
\pic{LAB2.png}{Cхема алгоритма расчета рангов}{rank}{H}
На рисунке \ref{gausstransform} представлена схема алгоритма преобразования матрицы СЛАУ к верхнетрапециедальному виду с полным переупорядовачиванием.
\pic{LAB3.png}{Cхема алгоритма преобразования матрицы СЛАУ}{gausstransform}{H}
На рисунке \ref{indexofmax} представлена схема алгоритма нахождения индексов максимального элемента в подматрице (i,i,n,n) основной матрицы СЛАУ $A_{NxN}$.
\pic{LAB4.png}{Cхема алгоритма нахождения индексов максимального элемента}{indexofmax}{H}
На рисунке \ref{ispositive} представлена схема алгоритма проверки матрицы на диагональное преобладание.
\pic{LAB5.png}{Cхема алгоритма проверки матрицы на диагональное преобладание}{ispositive}{H}
На рисунке \ref{atmatr} представлена схема алгоритма домножения расширенной матрицы СЛАУ на транспонированную основную.
\pic{LAB6.png}{Cхема алгоритма домножения расширенной матрицы СЛАУ}{atmatr}{H}
На рисунке \ref{zeidel} представлена схема алгоритма решения СЛАУ по методу Гаусса-Зейделя.
\pic{LAB7.png}{Cхема алгоритма решения СЛАУ по методу Гаусса-Зейделя}{zeidel}{H}
На рисунке \ref{zstep} представлена схема алогритма шага итерации по методу Гаусса-Зейделя.
\pic{LAB8.png}{Cхема алгоритма шага итерации по методу Гаусса-Зейделя}{zstep}{H}
%\vfill
\clearpage
\ssec{Инструкция пользователя}
Программа осуществляет построение аппроксимирующей функции тригонометрического базиса второй степени для функции $f(x)=x^4+3x-1$.

Для работе программе необходимы списки аргументов и значений функции. Аргументы функции добавляются в соотвествующий список в левой части окна; значения функции рассчитываются автоматически. Уже введенные значения можно удалять из списка. После этого нужно указать точность решения системы линейных уравнений и степень аппроксимирующей функции, и нажать кнопку "Посчитать". Программа построит графики аппроксимируемой и аппроксимирующей функции и выведет вид последней на форму. Также можно посчитать значения обеих функций от требуемого аргумента.

\ 
\ssec{Инструкция программиста}
При разработке программы построения аппроксимирующей функции были написаны следующие процедуры и функции:
%\suppressfloats[p]
\elist{
\item approx\_func - рассчитывает и возвращает значение члена аппроксимирующей функции
    на основе тригонометрического базиса.

approx\_func(n,x)

Параметры  функции \ftab{approxfunc:1}:
\tabl{Параметры  функции рассчета члена аппроксимирующей функции}{
\tabln{ n & целое & номер члена  аппроксимирующей функции}
\tabln{ x & веществ. & аргумент функции}
}{approxfunc:1}{H}

\item make\_matrix - функция получения расширенной матрицы коэфициентов
    для системы уравнений коэффициентов аппроксимирующей функции
    на основе тригонометрического базиса.
   
 Возвращает матрицу СЛАУ коэффициентов аппроксимирующей функции.

make\_matrix(x,y,n)

Параметры  функции \ftab{makematrix:1}:
\tabl{Параметры  функции получения расширенной матрицы коэфициентов}{
\tabln{ x,y & список & аргументы и значения аппроксимируемой функции}
\tabln{ n & целое & степень аппроксимирующей функции}
}{makematrix:1}{H}

Локальные переменные  функции \ftab{makematrix:2}:
\tabl{Локальные переменные  функции получения расширенной матрицы коэфициентов}{
\tabln{a & список & расширенная матрица СЛАУ коэффициентов функции}
\tabln{ i,j,k & целое & переменные-счетчики для работы со списками}
}{makematrix:2}{H}

\item make\_approx\_poly - расчет коэффициентов аппроксимирующей функции на основе тригонометрического базиса.

make\_approx\_poly(x,y,n,e)    

Параметры  функции \ftab{makeapproxpoly:1}:
\tabl{Параметры  функции расчета коэффициентов аппроксимирующей функции}{
\tabln{ x,y & список & аргументы и значения аппроксимируемой функции}
\tabln{ n & целое & степень аппроксимирующей функции}
\tabln{ e & веществ. & точность решения СЛАУ}
}{makeapproxpoly:1}{H}

Локальные переменные  функции \ftab{makeapproxpoly:2}:
\tabl{Локальные переменные  функции расчета коэффициентов аппроксимирующей функции}{
\tabln{ a & список & расширенная матрица СЛАУ коэффициентов функции}
}{makeapproxpoly:2}{H}
   

\item{Функция solvesys - функция решения СЛАУ.
Возвращает - решение и статистику о решении.
Если решение не найдено, возвращает None и
поясняющее сообщение.
	
	solvesys(a,e,iter\_check=True)

Параметры функции \ftab{solvesys:1}:
	\tabl{Параметры  функции решения СЛАУ}{
	\tabln{a &список& расширенная матрица СЛАУ}
	\tabln{e &веществ.& точность вычислений}
	\tabln{iter\_check &булев.&\parbox[t]{10cm}{флаг необходимости проверки сходимости на каждом шаге итераций}}
	}{solvesys:1}{H}
Локальные переменные функции \ftab{solvesys:2}:
	\tabl{Локальные переменные функции решения СЛАУ}{
	\tabln{at &список& \parbox[t]{10cm}{матрица системы, приведенная к трапец. виду
		(с полным переупорядовачением)}}
	\tabln{r,rext &целое& ранги основной и расширенной матриц СЛАУ}
	}{solvesys:2}{H}
}
\item Функция gauss\_transform - функция
    преобразования расширенной матрицы СЛАУ
    с полным переупорядовачиванием.
    Возвращает преобразованную матрицу.

        gauss\_transform(a)
        
Параметры функции \ftab{gausstransform:1}:
	\tabl{Параметры функции
    преобразования расширенной матрицы СЛАУ}{
    \tabln{a &список& расширенная матрица СЛАУ}
	}{gausstransform:1}{H}

Локальные переменные функции  \ftab{gausstransform:2}:
	\tabl{Локальные переменные  функции
    преобразования расширенной матрицы СЛАУ}{
    \tabln{at &список& копия матрицы а для преобразования.}
    \tabln{i,j,k &целое& переменные-счетчики для перебора элементов матрицы a.}
    \tabln{i2,j2 &целое& \parbox[t]{12cm}{координаты элемента для перестановки на диагональ
        матрицы}}
    \tabln{factor &веществ.& коэффициент домножения на слагаемую строку матрицы}
	}{gausstransform:2}{H}
\item index\_of\_max - функция  нахождения индексов максимального по модулю элемента 
в подматрице (i,i,n,n) в основной матрице СЛАУ A(n x n).
		
Внутренняя функция функции gauss\_transform.
			
			index\_of\_max()
		
Переменные объемлющей функции \ftab{indexofmax:1}:
\tabl{Переменные объемлющей функции функции  нахождения индексов максимального по модулю элемента}{
	\tabln{at &список& копия матрицы а для преобразования.}
	\tabln{i &целое& переменная-счетчик; кордината левого верхнего угла подматрицы}
	}{indexofmax:1}{H}
	
Локальные переменные функции \ftab{indexofmax:2}:
\tabl{Локальные переменные функции нахождения индексов максимального по модулю элемента}{
	\tabln{m &веществ.& модуль максимального элемента}
	\tabln{j,k &целое& переменные-счетчики для перебора элементов матрицы at.}
	\tabln{i2,j2 &целое& индексы максимального элемента}
	}{indexofmax:2}{H}
	
\item Функция is\_positive - функции проверки матрицы на диагональное преобладание.
    Если матрица имееет диагональное преобладание, возвращает True,
    иначе возвращает False.
	
		is\_positive(a)
		
Параметры функции  \ftab{ispositive:1}:
	\tabl{Параметры функции проверки матрицы на диагональное преобладание}{
    \tabln{a &список& расширенная матрица СЛАУ}
	}{ispositive:1}{H}

Локальные переменные функции  \ftab{ispositive:2}:
	\tabl{Локальные переменные функции проверки матрицы на диагональное преобладание}{
    \tabln{k &целое& количество преобладающих диагональных элементов}
    \tabln{s &веществ.& сумма модулей элементов строки без диагонального.}
    \tabln{i,j &целое& переменные-счетчики для перебора элементов матрицы a.}
	}{ispositive:2}{H}

\item rank - функция расчета рангов основной и расширенной матриц СЛАУ.
		
		rank(a)

Параметры функции  \ftab{rank:1}:
	\tabl{Параметры функции расчета рангов основной и расширенной матриц}{
    \tabln{a &список& расширенная матрица СЛАУ}
	}{rank:1}{H}

Локальные переменные функции  \ftab{rank:2}:
	\tabl{Локальные переменные функции расчета рангов основной и расширенной матриц}{
    \tabln{at &список& преобразованная в трапециедальную матрица а}
    \tabln{r,rext &целое& ранги основной и расширенной матриц СЛАУ}
    \tabln{i &целое& переменная-счетчик для перебора элементов матрицы a.}
	}{rank:2}{H}

\item at\_matr - функция домножения расширенной матрицы СЛАУ на транспонированную основную.
    
		at\_matr(a)
	
Параметры функции  \ftab{atmatr:1}:
	\tabl{Параметры функции домножения расширенной матрицы на транспонированную}{
    \tabln{a &список& расширенная матрица СЛАУ}
	}{atmatr:1}{H}

Локальные переменные функции  \ftab{atmatr:2}:
	\tabl{Локальные переменные функции домножения расширенной матрицы на транспонированную}{
    \tabln{an &список& преобразованная матрица СЛАУ}
    \tabln{i,j,k &целое& переменные-счетчики.}
	}{atmatr:2}{H}

\item norm - функция расчета нормы матрицы $\alpha$ для методов
    простых итераций и Гаусса-Зейделя.
    	
		norm(a)
	
Параметры функции  \ftab{norm:1}:
	\tabl{Параметры функции расчета нормы матрицы}{
    \tabln{a &список& расширенная матрица СЛАУ}
	}{norm:1}{H}

Локальные переменные функции  \ftab{norm:2}:
	\tabl{Локальные переменные функции расчета нормы матрицы}{
    \tabln{s &веществ.& сумма квадратов строки матрицы $\alpha$}
    \tabln{line &список& строка матрицы alpha}
    \tabln{a\_ij &веществ.& текущий элемент alpha} 
    \tabln{i,j &целое& переменные-счетчики.}
	}{norm:2}{H}

\item zeidel - функция решения СЛАУ по методу Гаусса-Зейделя
    Возвращает - решение и статистику о решении.
    Если решение не найдено, возвращает None и
    поясняющее сообщение.
	
		zeidel(a,e,iter\_check)
		
Параметры функции  \ftab{zeidel:1}:
	\tabl{Параметры функции решения СЛАУ по методу Гаусса-Зейделя}{
    \tabln{a &список& расширенная матрица СЛАУ}
    \tabln{e &веществ.& точность вычислений}
	\tabln{iter\_check &булев.&\parbox[t]{10cm}{ флаг необходимости проверки сходимости
		на каждом шаге итераций}}
	}{zeidel:1}{H}

Локальные переменные функции  \ftab{zeidel:2}:
	\tabl{Локальные переменные функции решения СЛАУ по методу Гаусса-Зейделя}{
	\tabln{x &список& решение СЛАУ}
	\tabln{stats &словарь& статистика решения}
	\tabln{n &веществ.& коэффициент оценки погрешности}
	\tabln{ek0,ek &веществ.& точность на предыдущем и текущем шаге}
	}{zeidel:2}{H}

\item z\_step - функция итерации метода Гаусса-Зейделя. Внутрення функция функции zeidel.
	Возвращает рассчитанное решение и точность этого решения.

		z\_step()
			
Переменные объемлющей функции(zeidel)  \ftab{zstep:1}:
	\tabl{Переменные объемлющей функции функции итерации метода Гаусса-Зейделя}{
	\tabln{a &список& расширенная матрица СЛАУ}
	\tabln{x &список& решение СЛАУ}
	\tabln{stats &словарь& статистика решения}
	}{zstep:1}{H}
Локальные переменные функции  \ftab{zstep:2}:
	\tabl{Локальные переменные функции итерации метода Гаусса-Зейделя}{
	\tabln{ek &веществ.& точность на текущем шаге}
	\tabln{i,j &целое& переменные-счетчики для перебора элементов матрицы a.}
	}{zstep:2}{H}
}
\clearpage
\ssec{Текст программы}
Реализиция задачи построения  интерполяционного многочлена Ньютона с разделенными разностями написана на языке Python 3.2 и состоит из трех частей.

Первая часть, файл mat3.py, содержит подпрограммы построения аппроксимирующей функции на основе тригонометрического базиса. Исходный текст этого модуля приводится ниже.

\prog{Python}{mat3.py}

Во второй части, представленной ниже, находятся функции решения СЛАУ.

\prog{Python}{mat1.py}

Третья часть - файл inter.py - является графическим интерфейсом для вычислительного ядра первых двух модулей. Далее представлен текст этого  модуля.

\prog{Python}{inter.py}

\clearpage

\ 
\ssec{Тестовый пример}
Ниже на рисунке \ref{scr:1} представлен пример работы программы при построении аппроксимирующей функции тригонометрического базиса второй степени для функции $f(x)=x^4+3x-1$.
\pic{SCR1.png}{Пример работы программы}{scr:1}{H}
\clearpage
\ssec{Вывод}
В этой лабораторной работе я изучил методы среднеквадратического приближения функций. Аппроксимация применяется, когда число узлов для построения превышает число известных параметров. Метод среднеквадратического приближения дает хорошие результаты - функции, построенные с помощью такого метода, достаточно приближены к исходной. Аппроксимация часто используется во многих областях науки и техники.
\end{document}
