\input template.tex
\begin{document}
\selectlanguage{russian}
\maketitle 6 {ОДНОШАГОВЫЕ МЕТОДЫ РЕШЕНИЯ ДИФФЕРЕНЦИАЛЬНЫХ УРАВНЕНИЙ}
\setcounter{page}{2}
\normalfont
\ssec{Цель работы}
Цель работы заключается в том, чтобы изучить различные методы решения дифференциальных уравнений и написать программу, реализующий один из таких методов.

\ 
\ssec{Задание на работу}
Решить дифференциальное уравнение методом Эйлера
 $$y'=\frac{k}{x^2}-py^2$$
с заданными начальными условиями $y(a)=g$ на отрезке $[a,b]$ c шагом $h$.

Вариант задания для тестового примера:
$$k=0.4,\ p=0.4;\ a=1,\ b=2,\ g=1;\ h=0.1$$
\ssec{Теоретическая справка}
Дифференциальные  уравнения  связаны  с  построением  моделей  динамики  (движения) 
объектов  исследования.  Они  описывают,  как  правило,  изменение  параметров  объектов  во 
времени  (хотя  могут  быть  и  другие  случаи).  Результатом  решения  дифференциальных 
уравнений являются функции, а не числа, как при решении алгебраических уравнений, поэтому 
они и более трудоемки. 

При  использовании  численных  методов  решение  дифференциальных  уравнений 
представляется в табличном виде, т.е. получается совокупность значений  $(X_n,Y_n)$. Решение 
носит шаговый характер, т.е. по одной или нескольким начальным точкам  $(X,Y)$  за один шаг 
находят  следующую  точку,  затем  следующую  и  т.д.  Решение  между  двумя  соседними 
значениями аргумента $h= X_{n+1}-X_n$ называется шагом.  

Однако  прежде  чем  обсуждать  методы  решения,  приведем  некоторые  сведения  из  курса 
дифференциальных уравнений. 

В зависимости от числа независимых переменных, дифференциальные уравнения делятся 
на  две  категории:  обыкновенные  дифференциальные  уравнения  \\
(ОДУ),  содержащие  одну 
независимую  переменную,  и  уравнения  с  частными  производными, содержащими  нес\-колько 
независимых переменных (например, в механике сплошных сред искомой функцией является 
плотность,  $t^0$ ,  напряжение  и  др.,  а  аргументами  -  координаты  рассматриваемой  точки  в 
пространстве и время). 
 
     Обыкновенные  дифференциальные  уравнения  могут  содержать  одну  или  нес\-колько 
производных от искомой функции  $y = f (x)$ и могут быть записаны в виде:   
                                           $$ F (x, y, y ',..., y^{(n)}) = 0,      \eqno{(1)}$$ 
где   x  – независимая переменная. 
     
 Наивысший  порядок  $(n)$  производной,  входящей  в  уравнение  (1)  называется 
порядком дифференциального уравнения. В частности\\ 
  $F(x, y, y') = 0$ - дифференциальное уравнение I порядка.\\ 
  $F(x, y, y', y'') = 0$ - дифференциальное уравнение II порядка.\\ 
В ряде случаев удается выразить старшую производную в явном виде\\   
$$ y'=f(x,y );   y''=f(x , y ,y ').$$ 

Такие  уравнения  называют  уравнениями,  разрешенными  относительно  старшей 
производной. 

Линейным  дифференциальным уравнением  называется  уравнение,  линейное  относительно 
искомой функции и еѐ производных.  

  Решением дифференциального уравнения (1) $ n$ -го порядка называется всякая функция 
$y=\phi(x)$, которая после ее подстановки в (1) превращает его в тождество. Решение ОДУ 
может быть общим и частным. 

Общее решение ОДУ n -го порядка содержит  n  произвольных постоянных \\
$C_1 , C_2 ,C_3,..., C_n $ , т.е. решение ОДУ имеет вид:  $y =\phi(x,C_1,C_2,...,C_n)$. 
         
Частное решение ОДУ получается из общего, если  произвольным постоянным задать 
определенные значения. 
  
Будем искать решение на ряде дискретных точек  $t_0,t _1,...,t_n$, удаленных друг от друга на 
расстоянии $h = t_{n+1}- t_n=const$ , в виде 
                                     
$$x(t_1)=x(t_0)+\int_{t_0}^{t_1}f(x,t)dt,$$
полученном путем интегрирования уравнения  dx=f(x,t)dt . 

Если принять, что на отрезке $[t0,t1] x' = x'(t_0) = f (x_0,t_0) = const$ , то 
$$ x(t_1)=x(t_0)+f(x,t)(t_1-t_0)$$
или, обозначив $t_1-t_0=h$, в дискретном виде  
$$x_1=x_0+x'_0h.$$
Для точки  $x_{n+1}$ можно записать  
                                             $$  x_{n+1}= x_n +x'_nh.  $$ 
Полученное выражение известно как явный (прямой) метод Эйлера. 

Искомая  функция  x(t)  на  шаге  интегрирования  была  аппроксимирована  прямой, 
совпадающей с касательной в точке  $x_n=x (t_n)$.  
 
В указанном выражении производная вычислялась в точке  $(x_0,t_0)$. Можно также выразить  $x_1$ 
через  $x_0$  и производную в точке $(x_1,t_1)$, т.е.  $x'_1 = f (x_1,t_1)$. Тогда  получим 
$$x_1=x_0+x'_1h$$
Или в общем виде 
                                          $$x_{n+1} =x_n+x'_{n+1} h $$ 
Эта формула называется неявным (обратным) методом Эйлера .
 
Последнюю формулу можно представить  в виде $x_{n+1} =x_n +f(x_{n+1},t_{n+1} )h$, где  $x_{n+1}$ входит и в правую 
часть .  Поэтому  эта  формула  пригодна,  когда  будет  предсказано  значение  $x_{n+1}$, 
например,  с  помощью  явного  метода  Эйлера.  Таким  образом,  мы  пришли  к  понятию 
«предсказание»,  когда  определяется  значение  искомой  функции  в  последующей  точке.  На 
основе  найденного  «предсказания»  можно  рассчитать  значение  $x'_{n+1}=  f ( x_{n+1},t_{n+1} )$ и 
использовать  его  при  коррекции,  которую  выполним  по  неявной  формуле  Эйлера.  Из-за 
ошибки «предсказания» может быть получена неточная коррекция. Чаще всего «предсказание» 
используется  в  качестве  начального  приближения  для  решения  уравнения  методом 
Ньютона. 
         
Еще  одну  формулу  численного  интегрирования  можно  получить,  приняв \\
$f(x,t)=\frac{1}{2}(x'_n+x'_{n+1})$, тогда:                                                                                        
$$x_{n+1}=x_n+\frac{1}{2}(x'_n+x'_{n+1}).$$

Это формула трапеции, которую иногда называют модифицированным методом Эйлера. 

Это  также  неявная  формула  интегрирования,  т.  к.  неизвестная  величина  $x'_{n+1}$  входит  в 
правую часть. Значение переменной  $x_{n+1}$ получают из решения нелинейного алгебраического 
уравнения  
$$x_{n+1}=x_n+\frac{1}{2}(f(x_n,t_n)+f(x_{n+1},t_{n+1})$$
методом Ньютона. 

Алгоритм неявного метода Эйлера отличается от алгоритма метода трапеции отсутствием в 
формуле определения  $x$  составляющей  $f (x_0,t_0)$  и вместо 
$\frac{1}{2}h$  используется $h$ . 

\ssec{Схема алгоритма}
На рисунке \ref{LAB1} представлена схема алгоритма расчета узлов и получения решения дифференциального уравнения по явному методу Эйлера.
\pic{LAB1.png}{Схема алгоритма  расчета узлов и получения решения уравнения}{LAB1}{H}
На рисунке \ref{LAB2} представлена схема алгоритма решения дифференциального уравнения по явному методу Эйлера.
\pic{LAB2.png}{Схема алгоритма решения уравнения по явному методу Эйлера}{LAB2}{H}

%\vfill
\clearpage
\ssec{Инструкция пользователя}
Программа позволяет решенить дифференциальное уравнение с пмощью явного метода Эйлера.

Программе необходимо передать границы диапазона поиска решения, шаг изменения аргумента в этом диапазоне, начальное условия - значение функции в нижней границе диапазона и параметры производной функции. Все данные вводятся в специально отведенные поля. Росле завершения ввода нажмите клавишу "Посчитать".

После завершения расчетов программа выведет на экран значение решения уравнения в заданных узлах.

\ 
\ssec{Инструкция программиста}
При разработке программы  решения дифференциального уравнения по явному методу Эйлера были написаны следующие функции:
%\suppressfloats[p]

\elist{
\item Функция euler - функция решения дифференциального уравнения по явному методу Эйлера.

Возвращает массив значений указанных узлов для функции-решения уравнения.

 euler::(Double->Double->Double)->[Double]->Double->[Double]

Параметры  функции \ftab{euler:1}:
\tabl{Параметры  функции решения дифференциального уравнения }{
\tabln{f & (Double->Double->Double) & производная искомой функции}
\tabln{x & [Double] & список узлов для поиска значений функции}
\tabln{y0 & Double & \parbox{10cm}{начальное условие – значение функции в первом узле.}}
}{euler:1}{H}

Локальные переменные  функции \ftab{euler:2}:
\tabl{Локальные переменные  функции  решения дифференциального уравнения}{
\tabln{euler' & [Double] & \parbox{10cm}{список значений решения уравнения в заданных узлах}}
\tabln{derivat & ([Double],[Double]) &\parbox{10cm}{ кортеж списка значений производной в узлах
    и шага между соседними узлами.}}
\tabln{deltas & [Double] & список шагов между соседними узлами.}
\tabln{derivates & [Double] & список значений производной в узлах.}
}{euler:2}{H}
}
\clearpage
\ssec{Текст программы}
Реализиция задачи  решения дифференциального уравнения по явному методу Эйлера написана на языке Haskell 98 и состоит из двух частей.

Первая часть, файл Euler.hs, содержит вычислительное ядро. Исходный текст этого модуля приводится ниже.

\prog{Haskell}{Euler.hs}

Вторая часть - файл Inter.hs - является графическим интерфейсом для вычислительного ядра первого модуля. Далее представлен текст этого второго модуля.

\prog{Haskell}{Inter.hs}

\clearpage

\ 
\ssec{Тестовый пример}
Ниже на рисунке \ref{SCR1} представлен пример работы программы при  решении дифференциального уравнения $y'=\frac{k}{x^2}-py^2$ с параметрами $k=0.4, p=0.4$ на отрезке $[a,b]=[1,2]$ c шагом $h=0.1$ и начальным условием $y(1)=1$ по явному методу Эйлера. 
\pic{SCR1.png}{Пример работы программы решения дифференциального уравнения}{SCR1}{H}
\clearpage
\ssec{Вывод}
В этой лабораторной работе я изучил различные одношаговые методы решения дифференциальных уравнений. Решение дифференциальных уравнений представляет важную задачу, так как широко применяется в различных областях науки и техники. Ручное решение может быть слишком долгим и трудоёмким, поэтому необходимо уметь применять численные методы решения дифференциальных уравнений с использованием вычислительной техники.
\end{document}
