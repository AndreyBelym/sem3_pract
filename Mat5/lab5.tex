\input template.tex
\begin{document}
\selectlanguage{russian}
\maketitle 5 {РЕШЕНИЕ НЕЛИНЕЙНЫХ УРАВНЕНИЙ (НУ) И СИСТЕМ 
НЕЛИНЕЙНЫХ УРАВНЕНИЙ}
\setcounter{page}{2}
\normalfont
\ssec{Цель работы}
Цель работы заключается в том, чтобы изучить различные методы решения нелинейных уравнений и систем нелинейных уравнений и написать программу, реализующий один из таких методов.

\ 
\ssec{Задание на работу}
Решить нелинейное уравнение методом хорд:
 $$\ln(\ln x)-e^{-x^2}=0$$

\ssec{Теоретическая справка}
Пусть дана некоторая функция  $f(x)$  и требуется найти все или некоторые значения $x$, для которых
$$f(x)=0 \eqno{(1)}$$

Значение  $x_*$, при котором $f(x_*)$ , называется корнем (или решением) уравнения (1).

Относительно функции $f(x)$  часто предполагается, что $f(x)$   дважды непрерывно дифференцируема в окрестности корня.

Корень   уравнения (1) называется простым, если первая производная функции $f(x)$  в точке  $x_*$ не равна нулю, т. е. $f'(x_*)\not=0$ . Если же  $f'(x_*)=0$, то корень  $x_*$  называется кратным корнем.

Геометрически корень уравнения (1) есть точка пересечения графика функции $f(x)$  с осью абсцисс.

Большинство методов решения уравнения (1) ориентировано на отыскание простых корней уравнения (1).

В процессе приближенного отыскания корней уравнения (1) обычно выделяют два этапа: локализация (или отделение) корня и уточнение корня.

Локализация корня заключается в определении отрезка $[a,b]$ , содержащего один и только один корень. Не существует универсального алгоритма локализации корня. В некоторых случаях отрезок локализации может быть найден из физических соображений. Иногда удобно бывает локализовать корень с помощью построения графика или таблицы значений функции $f(x)$ . На наличие корня на отрезке $[a,b]$  указывает различие знаков функции на концах отрезка.

Если функция   непрерывна на отрезке $[a,b]$   и принимает на его концах значения разных знаков, так, что $f(a)f(b)<0$ , то отрезок $[a,b]$  содержит, по крайней мере, один корень уравнения $f(x)=0$ .

Корень будет единственным, если производная $f'(x)$  существует на  $[a,b]$  и сохраняет на нем свой знак. 

На этапе уточнения корня вычисляют приближенное значение корня с заданной точностью $\varepsilon>0$ . Приближенное значение корня уточняют с помощью различных итерационных методов. Суть этих методов состоит в последовательном  вычислении  значений $x_0,x_1,...,x_n$, которые являются приближениями к корню $x_*$.

Метод хорд является модификацией метода Ньютона. Метод Ньютона требует для своей реализации вычисления производной, что ограничивает его применение. Метод секущих лишен этого недостатка. Если производную заменить ее приближением:
$$f'(x_n)\approx \frac{f(x_n)-f(x_{n-1})}{x_n-x_{n-1}}$$ ,
то вместо формулы метода Ньютона получим
 $$x_{n+1}=x_n - \frac{(x_n-x_{n-1})f(x_n)}{f(x_n)-f(x_{n-1})}                       \eqno{(2)}$$                                                      

Это означает, что касательные заменены секущими. Метод секущих является двушаговым методом, для вычисления приближения $x_{n+1}$   необходимо вычислить два предыдущих приближения $x_{n}$   и $x_{n-1}$ , и, в частности, на первой итерации надо знать два начальных значения $x_0$  и $x_1$  .

Формула (2) является расчетной формулой метода секущих.

Очередное приближение $x_{n+1}$  получается как точка пересечения с осью OX  секущей, соединяющей точки графика функции $f(x)$    с координатами $(x_{n-1},f(x_{n-1}))$ и  $(x_{n},f(x_{n}))$.

Пусть $x_*$   –  простой корень уравнения  (1) , и в некоторой окрестности этого корня функция  $f(x)$  дважды непрерывно дифференцируема, причем  $f''(x) \not=0$. Тогда найдется такая малая  окрестность корня , что при произвольном выборе начальных приближений    из этой окрестности итерационная последовательность, определенная по формуле (2) сходится.

Сравнение оценок  показывает, что  метод секущих сходится медленнее, чем метод Ньютона. Но в методе Ньютона на каждой итерации надо вычислять и функцию, и производную, а в методе секущих – только функцию. Поэтому при одинаковом объеме вычислений в методе секущих можно сделать примерно вдвое больше итераций и получить более высокую точность.

Так же, как и метод Ньютона, при неудачном выборе начальных приближений (вдали от корня) метод секущих может расходиться.  Кроме того применение метода секущих осложняется из-за того, что в знаменатель расчетной формулы метода (2) входит разность значений функции. Вблизи корня эта разность мала, и метод теряет устойчивость.

 Критерий окончания итераций метода секущих такой же, как и для метода Ньютона. При заданной точности   вычисления нужно вести до тех пор, пока не будет выполнено неравенство
 
$|x_n-x_{n-1}|<\varepsilon$                                     .                                               


\ssec{Схема алгоритма}
На рисунке \ref{LAB1} представлена схема алгоритма решения нелинейного уравнения методом хорд.
\pic{LAB1.png}{Схема алгоритма  решения нелинейного уравнения методом хорд}{LAB1}{H}
%\vfill
\clearpage
\ssec{Инструкция пользователя}
Программа позволяет решить нелинейное уравнение методом хорд.

Для работы введите параметры расчета (два начальных приближения корня, требуюмую точность и лимит итераций) в соотвествующие поля ввода на форме и нажмите кнопку "Посчитать". Программа выведет корень уравнения на форму.

\ 
\ssec{Инструкция программиста}
При разработке программы  решения нелинейных уравнений были написаны следующие  функции:
%\suppressfloats[p]
\elist{
\item chordSolver - рассчитывает список приближений корня функции.

chordSolver::(Double->Double)->[Double]->[Double]

Параметры  функции \ftab{chordSolver:1}:
\tabl{Параметры  функции расчета списка приближений корня}{
\tabln{f & (Double->Double) & функция для поиска корня}
\tabln{xs & [Double] & список приближений}
}{chordSolver:1}{H}

Локальные определения  функции \ftab{chordSolver:2}:
\tabl{Локальные определения  функции расчета списка приближений корня}{
\tabln{x & Double & новое вычисленное приближение}
}{chordSolver:2}{H}


\item firstWhenEpsOrCount - возвращает первый элемент списка, для которого 
-- выполнено условие точности либо некоторое количество итераций.

firstWhenEpsOrCount::(Double,Double)->[Double]->Double

Параметры  функции \ftab{firstWhenEpsOrCount:1}:
\tabl{Параметры  функции  выбора элемента списка}{
\tabln{xs & [Double] & список приближений}
\tabln{(eps,n) & (Double,Double) & значение точности и лимит итераций}
}{firstWhenEpsOrCount:1}{H}

\item chordSolve - расчет корня функции до выполнения условия точности либо до превышения лимита итераций.

chordSolve::(Double->Double)->(Double,Double)->(Double,Double)->Double

Параметры  функции \ftab{chordSolve:1}:
\tabl{Параметры  функции  расчета корня }{
\tabln{f & (Double->Double) & функция для поиска корня}
\tabln{(x0,x1) & (Double,Double) & начальные приближения}
\tabln{(eps,maxCount) & (Double,Double) & значение точности и лимит итераций}
}{chordSolve:1}{H}
}
\clearpage
\ssec{Текст программы}
Реализиция задачи  решения нелинейных уравнений написана на языке Haskell 98 и состоит из двух частей.

Первая часть является вычислительным ядром программы, текст этой части приводится ниже:

\prog{Haskell}{ChordSolve.hs}

Во втрой части содержится реализация графического интерфейса:

\prog{Haskell}{Inter.hs}

\clearpage

\ 
\ssec{Тестовый пример}
Ниже на рисунке \ref{SCR1} представлен пример работы программы при решении нелинейного уравнения $\ln(\ln x)-e^{-x^2}=0$ методом хорд.
\pic{SCR1.png}{Пример работы программы  решения нелинейных уравнений}{SCR1}{H}
\clearpage
\ssec{Вывод}
В этой лабораторной работе я изучил различные методы решения нелинейных уравнений и систем нелинейных уравнений. Так как многие уравнения в прикладных задачах невозможно привести к виду линейного уравнения ( или системы линейных уравнений), решение таких уравнений и систем представляет собой важную задачу в различных областях науки и техники.
\end{document}
